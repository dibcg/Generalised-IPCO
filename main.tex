  \documentclass[a4paper]{article}
  \usepackage[dvipsnames]{xcolor}
  \usepackage[inline]{enumitem}
  \usepackage{mathtools}
  \usepackage[utf8]{inputenc}
\usepackage[T1]{fontenc}
\usepackage[noadjust]{cite}
% \usepackage[english]{babel}
\usepackage{amsmath,amsthm,amssymb,amsfonts,caption,subcaption}
\captionsetup[subfigure]{labelformat=simple}

\usepackage{xspace}
\usepackage{hyperref}
\usepackage[nameinlink]{cleveref}
\usepackage{verbatim}
\usepackage{tikz}
\usepackage{setspace}
\usetikzlibrary{snakes}
\usepackage[procnumbered,linesnumbered,ruled,vlined]{algorithm2e}
\usetikzlibrary{decorations.pathmorphing}
\usetikzlibrary{shapes}
\hypersetup{
	colorlinks=true,breaklinks,
	linktoc=section,
	linkcolor=Maroon,
	linkbordercolor=white,
	citecolor=dartmouthgreen,
	urlcolor=dartmouthgreen,
	pdfborder = {0 0 1}
}


\usetikzlibrary{positioning, fadings, backgrounds}                

\usepackage[textsize=footnotesize,color=green!40]{todonotes}
\usepackage[hmargin=2.5cm,vmargin=3cm]{geometry}

\usetikzlibrary{decorations.pathmorphing}
\tikzset{snake it/.style={decorate, decoration=snake}}

\newenvironment{subproof}[1][\proofname]{%
  \renewcommand{\qedsymbol}{$\blacksquare$}%
  \begin{proof}[#1]%
}{%
  \end{proof}%
}


\newcommand{\IPC}{\textsc{Isometric Path Cover}\xspace}
\newcommand{\dist}[2]{\mathsf{d}\left(#1,#2\right)}
\newcommand{\distG}[3]{\mathsf{d}_{#1}\left(#2,#3\right)}



\newcommand{\subpath}[3]{#1\left( #2,#3 \right)}

\DeclareMathOperator{\close}{close}

\newcommand{\circumf}[3]{D_{#1}^{#2}\left(#3\right)}
%\newcommand{\ff}[1]{\textcolor{blue}{#1}}

\definecolor{dartmouthgreen}{rgb}{0.05, 0.5, 0.06}

\newtheorem{theorem}{Theorem}
\newtheorem{Question}{Question}
\newtheorem{proposition}{Proposition}
\newtheorem{lemma}{Lemma}
\newtheorem{observation}{Observation}
\newtheorem{corollary}{Corollary}
\newtheorem{notation}[theorem]{Notation}
\newtheorem{definition}{Definition}
\newtheorem{property}[theorem]{Property}
% \newtheorem{question}[theorem]{Question}
\newtheorem{conjecture}[theorem]{Conjecture}
\newtheorem{claim}{Claim}[theorem]
\newtheorem{problem}{Problem}
\newtheorem{remark}{Remark}

% \newtheorem{observation}[theorem]{Observation}
% \newcommand{\overbar}[1]{\mkern 1.5mu\overline{\mkern-1.5mu#1\mkern-1.5mu}\mkern 1.5mu}

\newcommand{\Pb}[4]{%
\begin{center}
  \begin{tabular}{|l|}%
  \hline
    \begin{minipage}[c]{0.95\textwidth}
      \smallskip%
      \par\noindent%
      #1%
      \par\noindent%
      %$\bullet$
      \textbf{\textsf{Input}}: #2% 
      \par\noindent%
      %$\bullet$
      \textbf{\textsf{#3}}: #4 
      \smallskip%
      \par\noindent%
    \end{minipage}
  \\\hline
  \end{tabular}%
\end{center}
}%


\newcommand{\ff}[1]{\textcolor{blue}{#1}}
\newcommand{\dd}[1]{\textcolor{red}{#1}}

\newcommand{\basegraph}[1]{X_{#1}}

\setstretch{1.1}
% \title{Complexity and algorithms for the Isometric Path Cover problem on chordal graphs and beyond}

  % \title{ Isometric path antichain covers: beyond hyperbolic graphs\thanks{This research was partially financed by the IFCAM project ``Applications of graph homomorphisms'' (MA/IFCAM/18/39), the ANR project GRALMECO (ANR-21-CE48-0004) and the French government IDEX-ISITE initiative 16-IDEX-0001 (CAP 20-25).}}

    \title{Discussion on Generalising IPCO, quasi isometric embedding}


  

%  \title{Complexity and algorithms for Isometric Path Cover on graphs with bounded treelength}

 \author{Dibyayan Chakraborty\footnote{School of Computer Science, University of Leeds, United Kingdom}
 \and Louis Esperet\footnote{GSCOP}}


\date{}
\begin{document}

\maketitle

\section{Directions}
 
 \newcommand{\quasipath}[2]{(#1,#2)\text{-quasi path}\xspace}
  \newcommand{\quasipaths}[2]{(#1,#2)\text{-quasi paths}\xspace}
  
 \begin{lemma}\label{lem:series-parallel}
 	Let $G$ be a $K_4$-minor free graph, $v$ be any vertex, and $P$ be any isometric path. Then there is an isometric subgraph $H$ of $G$ with bounded pathwidth that contains $P$ and $r$.
 \end{lemma}
 
 \begin{Question}
 	Is it possible to use \Cref{lem:series-parallel} to show that $K_4$-asymptotic minor free graphs admit a quasi-isometry with additive distortion on $K_4$-minor free graphs?
 \end{Question}
 
 For integers $a\geq 1,b\geq 0$, \emph{$(a,b)$-quasi-isometric embedding} of a graph $H$ in a graph $G$ is a map $f\colon V(G)\rightarrow V(H)$ such that for  $u,v \in V(H)$, $\frac{1}{a}\distG{H}{u}{v}-b\leq \distG{G}{f(a)}{f(b)} \leq a.\distG{H}{u}{v}+b$.
% A graph class $\cal G$ contains a graph $H$, as a \emph{quasi-isometric minor} if for all $a\geq 1,b\geq 0$ there is a graph $G\in \mathcal{G}$ such that there is a  $(a,b)$-quasi-isometric embedding of $H$ in $G$. 
 
%Not very clear. What I mean by the definitions.
 
 \begin{Question}
 	For an integers $a\geq 1,b\geq 0$ and $k\rightarrow\infty$, characterise graphs $G_k$ that does not contain a $(a,b)$-quasi-isometric embedding cycle of order $k$. 
 \end{Question}
 
 \todo{Correct the definitions below.}
 
  For integers $a\geq 1,b\geq 0$, an $\quasipath{a}{b}$ in $G$ is a graph $G$ that contains an $(a,b)$-quasi-isometric embedding of path. 
 
 \begin{definition}[Possible generalisations of ipco-1]\label{def:1}
 Strong quasi-isometric path complexity of a graph class $\mathcal{G}$ is the minimum integer $k$, such that for all $a\geq 1,b\geq 0$, there exist integers $A\geq 1,B\geq 0$ such that for all vertex $v\in V(G)$ and $\quasipath{a}{b}$ in $G$, there exists $k$-many $\quasipaths{A}{B}$ in $G$ containing $\{r\}\cup V(P)$. 
 \end{definition}
 
 \begin{definition}[Possible generalisations of ipco-2]\label{def:2}
 	``Strong quasi-isometric path cover width'' of a graph class $\mathcal{G}$ is the minimum integer $k$, such that for all $a\geq 1,b\geq 0$, there exist integers $A\geq 1,B\geq 0$ such that for all vertex $v\in V(G)$ and $\quasipath{a}{b}$ in $G$, there exists a graph $H$ in $G$ that contains $\{r\}\cup V(P)$ and admits an $(A,B)$-quasi-isometric embedding in a graph of pathwidth at most $k$. 
 \end{definition}
 
 \begin{Question}
 	Does \Cref{def:1} imply \Cref{def:2}?
 \end{Question}
\end{document}
