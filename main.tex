  \documentclass[a4paper]{article}
  \usepackage[dvipsnames]{xcolor}
  \usepackage[inline]{enumitem}
  \usepackage{mathtools}
  \usepackage[utf8]{inputenc}
\usepackage[T1]{fontenc}
\usepackage[noadjust]{cite}
% \usepackage[english]{babel}
\usepackage{amsmath,amsthm,amssymb,amsfonts,caption,subcaption}
\captionsetup[subfigure]{labelformat=simple}

\usepackage{xspace}
\usepackage{hyperref}
\usepackage[nameinlink]{cleveref}
\usepackage{verbatim}
\usepackage{tikz}
\usepackage{setspace}
\usetikzlibrary{snakes}
\usepackage[procnumbered,linesnumbered,ruled,vlined]{algorithm2e}
\usetikzlibrary{decorations.pathmorphing}
\usetikzlibrary{shapes}
\hypersetup{
	colorlinks=true,breaklinks,
	linktoc=section,
	linkcolor=Maroon,
	linkbordercolor=white,
	citecolor=dartmouthgreen,
	urlcolor=dartmouthgreen,
	pdfborder = {0 0 1}
}


\usetikzlibrary{positioning, fadings, backgrounds}                

\usepackage[textsize=footnotesize,color=green!40]{todonotes}
\usepackage[hmargin=2.5cm,vmargin=3cm]{geometry}

\usetikzlibrary{decorations.pathmorphing}
\tikzset{snake it/.style={decorate, decoration=snake}}

\newenvironment{subproof}[1][\proofname]{%
  \renewcommand{\qedsymbol}{$\blacksquare$}%
  \begin{proof}[#1]%
}{%
  \end{proof}%
}


\newcommand{\IPC}{\textsc{Isometric Path Cover}\xspace}
\newcommand{\dist}[2]{\mathsf{d}\left(#1,#2\right)}
\newcommand{\distG}[3]{\mathsf{d}_{#1}\left(#2,#3\right)}



\newcommand{\subpath}[3]{#1\left( #2,#3 \right)}

\DeclareMathOperator{\close}{close}

\newcommand{\circumf}[3]{D_{#1}^{#2}\left(#3\right)}
%\newcommand{\ff}[1]{\textcolor{blue}{#1}}

\definecolor{dartmouthgreen}{rgb}{0.05, 0.5, 0.06}

\newtheorem{theorem}{Theorem}
\newtheorem{Question}{Question}
\newtheorem{proposition}{Proposition}
\newtheorem{lemma}{Lemma}
\newtheorem{observation}{Observation}
\newtheorem{corollary}{Corollary}
\newtheorem{notation}[theorem]{Notation}
\newtheorem{definition}{Definition}
\newtheorem{property}[theorem]{Property}
% \newtheorem{question}[theorem]{Question}
\newtheorem{conjecture}[theorem]{Conjecture}
\newtheorem{claim}{Claim}[theorem]
\newtheorem{problem}{Problem}
\newtheorem{remark}{Remark}

% \newtheorem{observation}[theorem]{Observation}
% \newcommand{\overbar}[1]{\mkern 1.5mu\overline{\mkern-1.5mu#1\mkern-1.5mu}\mkern 1.5mu}

\newcommand{\Pb}[4]{%
\begin{center}
  \begin{tabular}{|l|}%
  \hline
    \begin{minipage}[c]{0.95\textwidth}
      \smallskip%
      \par\noindent%
      #1%
      \par\noindent%
      %$\bullet$
      \textbf{\textsf{Input}}: #2% 
      \par\noindent%
      %$\bullet$
      \textbf{\textsf{#3}}: #4 
      \smallskip%
      \par\noindent%
    \end{minipage}
  \\\hline
  \end{tabular}%
\end{center}
}%


\newcommand{\ff}[1]{\textcolor{blue}{#1}}
\newcommand{\dd}[1]{\textcolor{red}{#1}}

\newcommand{\basegraph}[1]{X_{#1}}

\setstretch{1.1}
% \title{Complexity and algorithms for the Isometric Path Cover problem on chordal graphs and beyond}

  % \title{ Isometric path antichain covers: beyond hyperbolic graphs\thanks{This research was partially financed by the IFCAM project ``Applications of graph homomorphisms'' (MA/IFCAM/18/39), the ANR project GRALMECO (ANR-21-CE48-0004) and the French government IDEX-ISITE initiative 16-IDEX-0001 (CAP 20-25).}}

    \title{Discussion on Generalising IPCO, quasi isometric embedding}


  

%  \title{Complexity and algorithms for Isometric Path Cover on graphs with bounded treelength}

 \author{Dibyayan Chakraborty\footnote{School of Computer Science, University of Leeds, United Kingdom}
 \and Louis Esperet\footnote{GSCOP}}


\date{}
\begin{document}

\maketitle

\section{Main definitions}

\begin{definition}
	
	An \emph{$(M,A)$-quasi-isometry} between graphs $G$ and $H$ is a map $f\colon V(G)\rightarrow V(H)$ such that the following holds for fixed constants $M\geq 1, A\geq 0$, \begin{enumerate*}[label=(\alph*)]
		\item $M^{-1}\dist{x}{y}-A  \leq \dist{f(x)}{f(y)} \leq M\dist{x}{y}+A$ for every $x,y\in V(G)$; and \item for every $z\in V(H)$ there is $x\in V(G)$ such that $\dist{z}{f(x)}\leq A$. 
	\end{enumerate*}
	The graphs $G$ and $H$ are \emph{quasi-isometric} if there exists a quasi-isometric map $f$ between $V(G)$ and $V(H)$.
\end{definition}


 \begin{definition}
	For integers $a\geq 1,b\geq 0$, \emph{$(a,b)$-quasi-isometric embedding} of a graph $H$ in a graph $G$ is a map $f\colon V(G)\rightarrow V(H)$ such that for  $u,v \in V(H)$, $\frac{1}{a}\distG{H}{u}{v}-b\leq \distG{G}{f(a)}{f(b)} \leq a.\distG{H}{u}{v}+b$.
\end{definition}



\section{Directions}
 
 \newcommand{\quasipath}[2]{(#1,#2)\text{-quasi path}\xspace}
  \newcommand{\quasipaths}[2]{(#1,#2)\text{-quasi paths}\xspace}
  
  
  
  
 \begin{lemma}\label{lem:series-parallel}
 There is a constant $c$ such that for any $K_4$-minor free graph $G$, $r\in V(G)$, and any isometric path $P$ of $G$, there is an isometric subgraph $H$ of $G$ with  pathwidth at most $c$ that contains $P$ and $r$. \todo{I am writing the proof which turned out to be a little more complicated than previously thought.}
 \end{lemma}
 
 
 
 
 \begin{Question}
 	Is it possible to use \Cref{lem:series-parallel} to show that $K_4$-asymptotic minor free graphs admit a quasi-isometry with additive distortion on $K_4$-minor free graphs?
 \end{Question}
 

% A graph class $\cal G$ contains a graph $H$, as a \emph{quasi-isometric minor} if for all $a\geq 1,b\geq 0$ there is a graph $G\in \mathcal{G}$ such that there is a  $(a,b)$-quasi-isometric embedding of $H$ in $G$. 
 
%Not very clear. What I mean by the definitions.
% 
% \begin{Question}\label{quest:forbid-cycle}
% 	For an integers $a\geq 1,b\geq 0$ and $k\rightarrow\infty$, characterise graphs $G_k$ that does not contain a $(a,b)$-quasi-isometric embedding cycle of order $k$.  \todo{Please confirm if this is the problem statemenet we are targeting now.}
% \end{Question}
  \begin{conjecture}\label{lem:cycle}
 	There is a function $f\colon \mathbb{N}\times \mathbb{N} \times \mathbb{N} \rightarrow \mathbb{N}$ such that a graph $G$  does not contain any $(a,b)$-quasi-isometric embedding of $C_k$ if and only if $G$ is quasi-isometric to a graph of girth at least $f(k,a,b)$.
 \end{conjecture}

 
 Is the above theorem anyway related to the following conjecture?
 
 \begin{conjecture}[\cite{albrechtsen2023structural}]
There is a function $f\colon \mathbb{N} \rightarrow \mathbb{N}$ such that if a connected graph $G$ does not contain a $c$-quasigeodesic $(\geq ck)$-subdivision of $K^3$ for some $c,k \in \mathbb{N}$, then $G$ has radial tree-width at most $f(k)$. \todo{radial tree-width = tree-length. I remember that we agreed that it is not related, but I want to convince myself again that it is unrelated.}
 \end{conjecture}


	What is the characterisation of a graph that does not contain $(a,b)$-quasi-isometric embedding of a tree $T$? Is the above question interesting? AIn general what is the characterisation of a graph that does not contain $(a,b)$-quasi-isometric embedding of a graph $H$ whsoe diameter is much larger than $a,b$? 
<<<<<<< HEAD

\begin{conjecture}\label{conj:too-general}
 For  integers $a,b$ and a graph $H$ with diameter $k>> \max\{a,b\}$, a graph $G$ does not contain $(a,b)$-quasi-isometric embedding of $H$ if and only if $G$ is quasi-isometric to an $H$-free graph. (in the induced sense) \todo{This is rater a question !! }
\end{conjecture} 
%=======
%
%\begin{conjecture}\label{conj:too-general}
% For  integers $a,b$ and a graph$H$ with diameter $k>> \max\{a,b\}$, a graph $G$ does not contain $(a,b)$-quasi-isometric embedding of $H$ if and only if $G$ is quasi-isometric to an $H$-free graph. (in the induced sense) \todo{This is rater a question !! }
%\end{conjecture} 
%>>>>>>> 2ae209c25c52435528f5d58e8eb48940aae93ca2
% \todo{Correct the definitions below.}
 
%For integers $a\geq 1,b\geq 0$, an $\quasipath{a}{b}$ in $G$ is a graph $G$ that contains an $(a,b)$-quasi-isometric embedding of path. 
 
% \begin{definition}[Possible generalisations of ipco-1]\label{def:1}
% Strong quasi-isometric path complexity of a graph class $\mathcal{G}$ is the minimum integer $k$, such that for all $a\geq 1,b\geq 0$, there exist integers $A\geq 1,B\geq 0$ such that for all vertex $v\in V(G)$ and $\quasipath{a}{b}$ in $G$, there exists $k$-many $\quasipaths{A}{B}$ in $G$ containing $\{r\}\cup V(P)$. 
% \end{definition}
% 
% \begin{definition}[Possible generalisations of ipco-2]\label{def:2}
% 	``Strong quasi-isometric path cover width'' of a graph class $\mathcal{G}$ is the minimum integer $k$, such that for all $a\geq 1,b\geq 0$, there exist integers $A\geq 1,B\geq 0$ such that for all vertex $v\in V(G)$ and $\quasipath{a}{b}$ in $G$, there exists a graph $H$ in $G$ that contains $\{r\}\cup V(P)$ and admits an $(A,B)$-quasi-isometric embedding in a graph of pathwidth at most $k$. 
% \end{definition}
% 
% \begin{Question}
% 	Does \Cref{def:1} imply \Cref{def:2}?
% \end{Question}
% 
 
% \begin{Question}
% 	Is there a function $f\colon \mathbb{N}\times \mathbb{N} \rightarrow \mathbb{N}$ such that any graph with degree at most $d$ and strong isometric path complexity at most $k$ have twin-width at most $f(k,d)$?
% \end{Question} 
% 
%For hyperbolic graphs with bounded degree, the answer is positive. For $K_{2,t}$-asymptotic minor-free graphs, the answer is positive. 
%

\newpage

\section{Preliminary lemmas}

\newcommand{\IndGraph}[2]{{#1}_{#2}}
For an integer $k\geq 2$ and a graph $G$, a set $S\subseteq V(G)$ is a \emph{distance-$k$ independent set} if for any $u,v\in S$, the distance between $u,v$ in $G$ is at least $k$. The graph $\IndGraph{G}{S}$ has vertex set $S$, and two vertices $u,v\in S$ are adjacent in $\IndGraph{G}{S}$ if $\distG{G}{u}{v}\leq 2k+1$. 

\begin{lemma}
Let $G$ be a graph, $k\geq 2$ be an integer, and $S$ be a maximum distance-$k$ independent set $S$ of $G$. Then, there is a $(2k+1,1)$-quasi-isometric map between $V(G)$ to $V(H)$ where $H\coloneqq \IndGraph{G}{S}$.
\end{lemma}
\begin{proof}
	For a vertex $v\in V(G)$, let $f(v)$ denote a vertex in $S$ which is closest to $v$ in $G$. Formally, $f(v)=z$ where $z\in S, \dist{G}{v}{z}=\min\{\distG{G}{v}{z'}\colon z'\in S\}$. We shall show that, $f$ is the desired map. First notice that, for any vertex $v\in V(G)$, $\distG{G}{v}{f(v)} \leq k$. Otherwise, $S\cup \{v\}$ is a larger distance-$k$-independent set. Hence, $\distG{G}{u}{v}\leq (2k+1)\cdot \distG{H}{f(u)}{f(v)} + 2k$. 
	
	Now consider any edge $uv\in E(G)$ such that $f(u)\not= f(v)$. Observe that, $\distG{G}{f(u)}{f(v)}\leq \distG{G}{f(u)}{u} + \distG{G}{f(v)}{v} + 1$. This implies, $\distG{G}{f(u)}{f(v)}\leq 2k+1$, and therefore $f(u)$ must be adjacent to $f(v)$ in $H$. 
	Now, consider any isometric path $P$ in $G$  and let $P\coloneqq u_1u_2\ldots u_t$. Above arguments imply, $W\coloneqq f(u_1)f(u_2)\ldots f(u_t)$ is a walk in $H$ consisting of at most $t$ many distinct vertices and therefore, $\distG{H}{f(u_1)}{f(u_t)}\leq \distG{G}{u_1}{u_t}$. Hence, for any two vertices $u,v\in V(G)$ we have $\frac{1}{2k+1}\distG{G}{u}{v}-1 \leq \distG{H}{f(u)}{f(v)} \leq \distG{G}{u}{v}$.
\end{proof}


\section{Proof for \Cref{lem:cycle}}

We prove this theorem in the following two steps. 

\begin{lemma}
For integers $a,b,k$ and a graph $G$ with no $(a,b)$-quasi-isometric embedding of $C_k$,  there is an integer $k'$ and a graph $H$ such that the girth of $H$ is at least $k'$ and $G$ is quasi-isometric to $H$.
\end{lemma}

\begin{proof}
	For this direction use distance independent set. 
	
	\begin{itemize}
		\item Consider a maximum distance $2a$-independent set $S$.
		 
		 \item Observe that the  vertices in $S$ are close to each other in $G$. Use this to create $H$.
		 
		 \item Now map the vertices of $G$ to $H$ to its closest neighbor in $H$. 
		 
		 \item Observe that it is indeed a quasi-isometry.
		 
		 \item If $H$ has girth comparable to $k$, then $G$ has an $(a,b)$-quasi-isometric embedding of $C_k$, a contradiction.
	\end{itemize}
\end{proof}


\begin{lemma}
	For integers $a,b,k$ there is an integer $k'$ such that if a graph $G$ is quasi-isometric to a graph $H$ with girth at least $k'$, then $G$ does not contain $(a,b)$-quasi-isometric embedding of $C_k$. 
\end{lemma}

\begin{proof}
	Graphs with girth sufficiently larger than $k,a,b$, does not contain an $(a,b)$-isometric embedding of a cycle of length $k$. 
	The proof could use the fact that locally the target looks like a tree. Since cycles are not quasi-isometric to a tree, we arrive at a contradiction.
	
	\begin{itemize}
		
		\item Suppose for contradiction that $G$ contains $(a,b)$-quasi-isometric embedding of $C_k$.  First observe that the $C_k$ must be mapped to a subgraph $S$ of $G$ whose diameter depends on $k,a,b$. In fact, we should probably shopw that it will contain a cycle metric of large diameter.  Define $k'$ to be much larger than this diameter. 
		
		\item Since $S$ has small diameter, it must be mapped within a bounded radius ball in $H$.
		  
		\item Consider a vertex $v$ of $H$ and a ball $B_v$ around $v$ of radius that depends appropriately on $k,a,b$. 
		
		\item Since $k'$ is much larger, the subgraph $H'$ of $H$ induced by $B_v$ is a tree. 
		
		\item Let us look at the vertices of $G$ which are mapped to close to the vertices of $B_v$. Let us call this set $S$.
		
		\item Define $k'$ in such a way that we can infer the metric $(S,G)$ to be a tree metric and hence cannot embed a cycle metric of large diameter with small distortion.
		
		\item Adjust the quasi-isometry between $G$ and $H$ appropriately so that the distortion.
		\item This contradicts first item. 
	\end{itemize}
\end{proof}


Will it be wrong to follow the same strategy for answering \Cref{conj:too-general}?

\newpage 

\section{Proof of \Cref{lem:series-parallel}}


For the entirety of this section, let $G$ be a $K_4$-minor free graph, $r$ be a vertex of $G$, and $P$ be an isometric path of $G$. Let $\mathcal{P}$ be a minimum cardinality set of $r$-rooted isometric paths in $G$ that covers $P$ and for any $Q\in \mathcal{P}$, $|V(P)\cap V(Q)|$ is maximized. Let $H$ be the graph induced by the set $\displaystyle\bigcup\limits_{Q\in \mathcal{P}} V(Q)$. First, we prove the following lemma.

\begin{lemma}\label{lem:1st-isometric}
	If $|\mathcal{Q}| \geq 5$, then $H$ is isometric.
\end{lemma}
\begin{proof}
	Suppose $H$ is not isometric. Then, there exists $u,v\in V(H)$ such that $\distG{H}{u}{v} \neq \distG{G}{u}{v}$. Let $Q_u,Q_v \in \mathcal{Q}$ be isometric paths that contain $u$ and $v$, respectively. Since both $Q_u$ and $Q_v$ are isometric, it follows that $Q_u\neq Q_v$. Observe that, there must exist an $(u,v)$-isometric path $P$ in $G$ such that $V(P)\cap V(H)=\{u,v\}$ and $E(H)\cap E(P)=\emptyset$. \todo{Complete}
\end{proof}

Let $\mathcal{A}$ be the set of pairs of vertices in $H$ whose distance in $H$ is strictly greater than in $G$. Formally, $\mathcal{A}=\{(u,v) \colon \distG{H}{u}{v} \neq \distG{G}{u}{v}  \}$.  For each pair $(u,v)\in \mathcal{A}$, consider an $(u,v)$-isometric path $P_{uv}$ and put in $\mathcal{B}$. Let $H^*$ be the subgraph of $G$ induced by $V(H)\cup \displaystyle\bigcup\limits_{Q'\in \mathcal{B}} V(Q')$.  By \Cref{lem:1st-isometric}, if $|\mathcal{Q}|\geq 5$, then $H=H^*$. We prove the following lemma.

\begin{lemma}\label{lem:2nd-isometric}
	The graph $H^*$ is isometric.
\end{lemma}

Now we prove an upper bound on the pathwidth of $H^*$.

\begin{lemma}
	The pathwidth of $H^*$ is at most $c$. \todo{I thought that the pathwidth is $2$, but not sure anymore. It should be constant though. }
\end{lemma}
\begin{proof}
	Forbid a fixed binary tree as a minor.
\end{proof}
\bibliographystyle{plain}
\bibliography{references}
\end{document}
